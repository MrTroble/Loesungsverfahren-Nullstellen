\documentclass[11pt,a4paper]{article}
\usepackage[utf8]{inputenc}
\usepackage{amsmath}
\usepackage{amsfonts}
\usepackage{amssymb}
\usepackage{graphicx}
\begin{document}
\begin{center}
\huge \textbf{Kurvendiskussion}
\end{center}

\section{Allgemein}

Bei der Kurvendiskussion geht es Primär darum eine Funktion zu beschreiben. Dabei müssen Folgende Informationen ermittelt werden.

\begin{itemize}
\item Achsenschnittpunkte (x, y)
\item Extremata
\item Wendepunkte
\item Symmetrie
\end{itemize}

Des weiteren können folgende Sachen Abgefragt werden:

\begin{itemize}
\item Monotonie
\item Krümmung
\end{itemize}

Für alle Berechnungen brauchen wir die Ableitungen. Daher ist es ratsam die ersten drei Ableitungen direkt am Anfang zu berechnen und hin zu schreiben.

\section{Achsenschnittpunkte}

Wiederholung: Die X-Achsenschnittpunkte sind die Nullstellen der original Funktion. Zur Lösung von Nullstellen siehe des Skript der letzten Woche. Um den Y-Achsenschnittpunkt aus zu rechnen wird für x einfach 0 eingesetzt.

\newpage

\section{Extremata}

Um nun ein Extrempunkt aus zu rechnen muss man die Nullstellen der ersten Ableitung $f'(x)$ ausrechnen. Denn $f'(x)$ beschreibt die locale Änderung einer Funktion wenn diese Änderung (also die Tangentensteigung bzw $f'(x)$) also 0 ist dann kann es einen Extrempunkt geben. Um nun aber fest zu stellen ob es ein Hoch oder Tiefpunkt ist brauchen wir die zweite Ableitung $f''(x)$. Daher nimmt man nun die Ausgerechneten Nullstellen und setzt diese in $f''(x)$ wenn das Ergebnis größer null Handelt es sich um einen Tiefpunkt wenn das Ergebnis kleiner null einen Hochpunkt.
\newline
\newline
TL;DR
\begin{itemize}
\item $f'(x) = 0$
\item Tiefpunkt $f''(x) > 0$
\item Hochpunkt $f''(x) < 0$
\item Kein Extrempunkt $f''(x) = 0$
\end{itemize}

\section{Wendepunkt (Sattelpunkt)}
Beim Wendepunkt ist es ähnlich nur muss man hier die Nullstellen der zweiten Ableitung Ausrechnen und in die dritte Ableitung einsetzen. Wenn $f'''(x) \neq 0$ gibt es hier einen Wendepunkt.
\newline
\newline
TL;DR
\begin{itemize}
\item $f''(x) = 0$
\item Wendepunkt $f'''(x) \neq 0$
\item Kein Wendepunkt $f'''(x) = 0$
\end{itemize}

\section{Symmetrie}
Die Symmetrie lässt sich meist schon anhand der Exponenten erkennen. Gibt es nur Gerade Exponenten so ist die Funktion Achsen-symmetrisch, gibt es nur ungerade so ist sie Punkt-symmetrisch, sind die Exponenten gemischt so gibt es keine Symmetrie.

\newpage

\section{Aufgaben}

Analysieren sie folgende Funktionen. Erstellen sie auf Basis ihrer Analyse ein Schaubild.

\begin{eqnarray}
f(x) = 4x^2 + 12x\\
f(x) = 4x^2 + 12\\
f(x) = 4x^3 + 12x + 16\\
f(x) = 53x^5 + 21x
\end{eqnarray}

\newpage
\section{Lösungen}

\subsection{$f(x) = 4x^2 + 12x$}
Errechnung der Nullstellen
\begin{eqnarray}
4x^2 + 12x = 0\\
x(4x + 12) = 0 \rightarrow x_0 = 0\\
4x + 12 = 0\\
4x = -12\\
x_1 = -3
\end{eqnarray}
N(0\textbar 0) und N(-3\textbar 0) Y-Achsenschnittpunkt S(0\textbar 0)
\newline
Ab hier brauchen wir die Ableitungen.
\begin{eqnarray}
f'(x) = 8x + 12\\
f''(x) = 8
\end{eqnarray}

Errechnung der Extremata
\begin{eqnarray}
8x + 12 = 0\\
8x = -12\\
x = -\dfrac{3}{2}\\
f''(-\dfrac{3}{2}) > 0 \rightarrow \text{Tiefpunkt}\\
4(-\dfrac{3}{2})^2 + 12(-\dfrac{3}{2})\\
y = -9\\
\end{eqnarray}
T($-\dfrac{3}{2}$\textbar -9)
\newline
Keine Wendepunkte da $f''(x) \neq 0$
\newpage

\subsection{$f(x) = 4x^2 + 12$}
Nullstellen
\begin{eqnarray}
4x^2 + 12 = 0\\
4x^2 = -12\\
x^2 = -3\\
x = \sqrt{-3} \rightarrow \text{Keine Nullstellen}\\
f(0) = 12
\end{eqnarray}
Y-Achsenschnittpunkt S(0\textbar 12)

\begin{eqnarray}
f'(x) = 8x
f''(x) = 8
\end{eqnarray}

Extremata
\begin{eqnarray}
8x = 0 \rightarrow x = 0\\
f''(x) > 0 \rightarrow \text{Tiefpunkt}\\
f(0) = 12
\end{eqnarray}
T(0\textbar 12)
\newpage

\subsection{$f(x) = 4x^3 + 12x + 16$}

Nullstellen mit der lustigen Polynomdivision (geratene Nullstelle -1)
\begin{eqnarray}
4x^3 + 0x^2 + 12x + 16 : (x + 1) = 4x^2 - 4x + 16\\
4x^3 + 4x^2\qquad \qquad \qquad \qquad\qquad\qquad\qquad\qquad \\
-4x^2 + 12x \qquad \qquad\qquad\qquad\qquad\qquad\qquad\\
-4x^2 - 4x \qquad \qquad\qquad\qquad\qquad\qquad\qquad\\
16x + 16\qquad\qquad\qquad\qquad\qquad\qquad\\
16x + 16\qquad\qquad\qquad\qquad\qquad\qquad
\end{eqnarray}

\begin{eqnarray}
4x^2 - 4x + 16 = 0\\
x^2 - x + 4 = 0\\
\dfrac{1}{2} \pm \sqrt{\dfrac{1}{4} - 4} \rightarrow \text{Keine weitere Lösung}
\end{eqnarray}

S(-1\textbar 0) Y-Achsenschnittpunkt S(0\textbar 16)

Ableitungen
\begin{eqnarray}
f'(x) = 12x^2 + 12
f''(x) = 24x
f'''(x) = 24
\end{eqnarray}

Extremata
\begin{eqnarray}
12x^2 + 12 = 0\\
12x^2 = -12\\
x^2 = -1 \rightarrow \sqrt{-1} \text{Keine Extremata}
\end{eqnarray}

Wendepunkt
\begin{eqnarray}
24x = 0\\
x = 0\\
f'''(0) \neq 0 \rightarrow \text{Wendepunkt an x = 0}
\end{eqnarray}

W(0\textbar 16)

\end{document}